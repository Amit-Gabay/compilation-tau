%%%%%%%%%%%%%%%%%%%%%%%%%%
% document class: beamer %
%%%%%%%%%%%%%%%%%%%%%%%%%%
\documentclass{beamer}

%%%%%%%%%%%%
% PACKAGES %
%%%%%%%%%%%%
\usepackage{hyperref}
\usepackage{graphicx}
\usepackage{multirow}
\hypersetup{colorlinks=true,urlcolor=blue}

%%%%%%%%%%%%%%%%%%%%%%%%%%%%%
% no navigation barsplease %
%%%%%%%%%%%%%%%%%%%%%%%%%%%%%
\setbeamertemplate{footline}[frame number]{}

%%%%%%%%%%%%%%%%%%%%%%%%%%%%%%%%
% no navigation symbols please %
%%%%%%%%%%%%%%%%%%%%%%%%%%%%%%%%
\setbeamertemplate{navigation symbols}{}

%%%%%%%%%%%%%%%%%%%%%%%%%%%%
% no footers at all please %
%%%%%%%%%%%%%%%%%%%%%%%%%%%%
\setbeamertemplate{footline}{}

%%%%%%%%%%%%%%%%%%%%%%%%%%%
% center the title please %
%%%%%%%%%%%%%%%%%%%%%%%%%%%
\setbeamertemplate{frametitle}[default][center]

%%%%%%%%%%%%%%%%%%
% begin document %
%%%%%%%%%%%%%%%%%%
\begin{document}

%%%%%%%%%
% title %
%%%%%%%%%
\title{Compilation First Step: Lexical Analysis}

%%%%%%%%%%
% author %
%%%%%%%%%%
\author{Verifying that every single word is legal}

%%%%%%%%%%%%
% date ... %
%%%%%%%%%%%%
\date{\today} 

%%%%%%%%%%%%%
% frame ... %
%%%%%%%%%%%%%
\frame{\titlepage} 

%\frame{\frametitle{Table of contents}\tableofcontents} 

%%%%%%%%%%%%%%%%%%%%%%%%%%%%%%%
% SECTION :: Warm up examples %
%%%%%%%%%%%%%%%%%%%%%%%%%%%%%%%
\section{Warm up examples} 
%%%%%%%%%%%%%%%%%%%%%%%%%%%%%%%%%%
% Frame Open :: Warm up examples %
%%%%%%%%%%%%%%%%%%%%%%%%%%%%%%%%%%
\frame{\frametitle{Warm up examples}
%%%%%%%%%%%%%%%%%%
% Table :: Begin %
%%%%%%%%%%%%%%%%%%
\begin{table}[h]
\centering
\begin{tabular}{ | l | c | l | }
\hline
%%%%%%%%%%%%%%%%%%%%%%%%%%%%%%%%%%%
code & compilation line & status \\
%%%%%%%%%%%%%%%%%%%%%%%%%%%%%%%%%%%
\hline
& &\\
%%%%%%%%%%%%%%%%%%%%%%%%%%%%%%%%%%%%%%%%%%%%%%%%%%%%%%%%%%%%%%%%
int main()\{ 6; \} & gcc -Wall example.c & example.c:3:5: warning:\\
& & statement with no effect \\
& & 6; \\
& & \^ \\

%%%%%%%%%%%%%%%%%%%%%%%%%%%%%%%%%%%%%%%%%%%%%%%%%
& &\\
\hline
d & e & f \\ \hline
g & h & i \\ \hline
\end{tabular}
\caption{
abtrsact state
\label{Table_Examples_01_02}}
\end{table}


%%%%%%%%%%%%%%%%%%%%%%%%%%%%%%%%%%%
% Frame Close :: Warm up examples %
%%%%%%%%%%%%%%%%%%%%%%%%%%%%%%%%%%%
}

%%%%%%%%%%%%%%%%%%%%%%%%%%%%%%%%%%%%%%%%%%%%%%%%%%%%%%
% SECTION :: Implementation (High Level) Description %
%%%%%%%%%%%%%%%%%%%%%%%%%%%%%%%%%%%%%%%%%%%%%%%%%%%%%%
\section{Implementation (High Level) Description} 
%%%%%%%%%%%%%%%%%%%%%%%%%%%%%%%%%%%%%%%%%%%%%%%%%%%%%%%%%%
% Frame Title :: Implementation (High Level) Description %
%%%%%%%%%%%%%%%%%%%%%%%%%%%%%%%%%%%%%%%%%%%%%%%%%%%%%%%%%%
\frame{\frametitle{Implementation (High Level) Description}
%%%%%%%%%%%%%%%%%%
% Items :: Begin %
%%%%%%%%%%%%%%%%%%
\begin{itemize}
\item
Resolving occurrences of program string variables is
not trivial in C, which is weakly typed, and allows
taking the memory location of stored variables:
msg$\rightarrow$contents = (void *) $\&$buf;
\item
We use the term abstract buffers to denote the corresponding solver string variables.
As the program modifies strings with writes, strcpy etc.
abstrct buffers accumulate these changes by keeping consecutive
versions for abstract buffers.
\item
Using a many sorted solver inevitably introduces sort conversion issues.
Suppose that a bitvector is added to an integer:
(i $<<$ 5) + strlen( msg )
and it is sound to do this addition in both bitvector and integer domains.
Which expression should be converted? we say: whichever is faster!
%%%%%%%%%%%%%%%%
% Items :: End %
%%%%%%%%%%%%%%%%
\end{itemize}
}

%%%%%%%%%%%%%%%%%%%%%%%%%%%%
% SECTION :: Accelerations %
%%%%%%%%%%%%%%%%%%%%%%%%%%%%
\section{Accelerations} 
%%%%%%%%%%%%%%%%%%%%%%%%%%%%%%%%
% Frame Title :: Accelerations %
%%%%%%%%%%%%%%%%%%%%%%%%%%%%%%%%
\frame{\frametitle{Accelerations List}
%%%%%%%%%%%%%%%%%%
% Items :: Begin %
%%%%%%%%%%%%%%%%%%
\begin{itemize}
\item
Context Aware Sorts
\begin{itemize}
\item ((*s) == 'a') vs. ((*s) $<<$ 5)
\end{itemize}
\item
Tailored String Semantics
\begin{itemize}
\item int strcmp(s1,s2) $\rightarrow$ bool strcmp(s1,s2)
\item char *strchr(s,c) $\rightarrow$ bool strchr(s,c)
\end{itemize}
\item
Solver Performance Driven Query Rewriting
\begin{itemize}
\item str.indexof $\rightarrow$ str.contains
\item str.indexof $\rightarrow$ str.len
\item automatic deducing of query invariants
\end{itemize}
\item
Reducing Number of States with C to C translations:\\
char *f(char *d,char *s)
$\{$
while (*d++ = *s++);
$\}$
return d;
$\rightarrow$ strcpy(d,s); return d+strlen(src);
\item
Caching reads/writes.
\item
Reducing number of generated abtract buffer versions.
\item
Under approximation of programs paths.
For example, Ignoring toUpper and adding relevant constraints.
%%%%%%%%%%%%%%%%
% Items :: End %
%%%%%%%%%%%%%%%%
\end{itemize}
}

%%%%%%%%%%%%%%%%%%%%%%%%%%%%
% SECTION :: Accelerations %
%%%%%%%%%%%%%%%%%%%%%%%%%%%%
\section{Accelerations} 
%%%%%%%%%%%%%%%%%%%%%%%%%%%%%%%%
% Frame Title :: Accelerations %
%%%%%%%%%%%%%%%%%%%%%%%%%%%%%%%%
\frame{\frametitle{Accelerations :: Context Aware Sorts}
%%%%%%%%%%%%%%%%%%
% Items :: Begin %
%%%%%%%%%%%%%%%%%%
\begin{itemize}
\item
Sort conversions are expansive, and in some cases they can be avoided altogether.
Think of the following examples:\\
~ ~ ~ ~if ((*s) == ' ')           $\{$ s++; $\}$ \\
~ ~ ~ ~if (((*s) $<<$ 3) $<$ 100) $\{$ s++; $\}$ \\
Since the returned value from (*s) has a string sort,
then in order to shift it left, it needs to be converted to a bit vector.
In constrast, if it is simply compared to the white-space character,
then it needs not be converted at all.
\item
A simple pre processing of the program can easily identify locations
where such conversions are not needed.
%%%%%%%%%%%%%%%%
% Items :: End %
%%%%%%%%%%%%%%%%
\end{itemize}
}

%%%%%%%%%%%%%%%%%%%%%%%%%%%%
% SECTION :: Accelerations %
%%%%%%%%%%%%%%%%%%%%%%%%%%%%
\section{Accelerations} 
%%%%%%%%%%%%%%%%%%%%%%%%%%%%%%%%
% Frame Title :: Accelerations %
%%%%%%%%%%%%%%%%%%%%%%%%%%%%%%%%
\frame{\frametitle{Accelerations :: Tailored String Semantics}
%%%%%%%%%%%%%%%%%%
% Items :: Begin %
%%%%%%%%%%%%%%%%%%
\begin{itemize}
\item
The semantics of string library functions often contains more
details then actually needed by users.
\item
For example, whenever two strings are compared with strcmp(s1,s2),
the returned value is either $0$ when they are identical,
or the ascii difference between the first place of change.
\item
Usually users just ask whether the result is $0$ or not.
This gives a chance for a sound optimization, since it enables us
to use the native returned boolean sort from the solver
\item
Similarly, it is quite usual to use strchr to check the existence of
a certain character in a string:\\
if (strchr( s, ' ')) $\{$ return -1; $\}$\\
This enables a faster query.
%%%%%%%%%%%%%%%%
% Items :: End %
%%%%%%%%%%%%%%%%
\end{itemize}
}

%%%%%%%%%%%%%%%%%%%%%%%%%%%%
% SECTION :: Accelerations %
%%%%%%%%%%%%%%%%%%%%%%%%%%%%
\section{Accelerations} 
%%%%%%%%%%%%%%%%%%%%%%%%%%%%%%%%
% Frame Title :: Accelerations %
%%%%%%%%%%%%%%%%%%%%%%%%%%%%%%%%
\frame{\frametitle{Accelerations :: Query Rewriting}
%%%%%%%%%%%%%%%%%%
% Items :: Begin %
%%%%%%%%%%%%%%%%%%
\begin{itemize}
\item
Some queries happen often, like boundaries checks before each strcpy operation.
The solver tries to figure out whether the first null termination of the source
can be larger than the destination's size. As it so happens, sometimes the
source buffer has no other constraints, and so clearly the null termination 
can be the very last byte. In these cases, replacing str.indexof with str.len
is much faster.
\item
In the strchr case described before, str.indexof can be replaced with str.contains
which is much faster.
\item
Some queries deal with very large ($>$ 1000 bytes long) strings.
As the solver struggles with the huge size,
it is sometimes possible to automatically prove an equivalent satisfiable query
with much shorter sizes, and infer back the model from the small string
to the large original string
%%%%%%%%%%%%%%%%
% Items :: End %
%%%%%%%%%%%%%%%%
\end{itemize}
}

\frame{\frametitle{Implementation :: Details}
\begin{itemize}
\item Keep KLEE's memory model: Each allocation has a MemoryObject.
\item Use $markString$ calls to tag memory objects which we want to consider as strings.
\item Writes to strings get translated to $version_{i+1} = prefixVersion_i ++ newChar ++ suffixVersion_i$ 
\item For each read we introduce a new BitVectorVariable, that is return of the read. Then we make a string out of it and assert it is equal to the desired read location
\item Allocations of symbolic sizes are currently conretize to a value between 10 and 128. (Easy to change this)
\item Currently support $strcmp$, $str(n)cpy$, $strchr$ and $strlen$.
\item Calls to these intercepted in SpecialFunctionHandler, arguments resolved to MemoryObjects and through them to AbstractBuffers. Then appropriate constraints are added to the state.
\end{itemize}
}

\frame{\frametitle{Implementation :: Integration with KLEE}
\begin{itemize}
\item KLEE is very bitvector orientated: expressions need witdh, and symbolic things can only be Arrays of bitvectors.
\item Width is important to corectly interface with Extract and Concat expression and Casting.
\item Arrays are important to correctly interface with the solver and test case generation
\item Currently we set the width of all Strings/Ints to 8 (se they fit in1 memory cell) and special case on type (we started to type the Expression language)
\item Arrays for Strings are represented as 0 length Arrays of bitvectors (seems to work well enough so far)
\end{itemize}
}

\end{document}
