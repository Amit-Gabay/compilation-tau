%%%%%%%%%%%%%%%%%%%%%%%%%%
% SECTION :: IR examples %
%%%%%%%%%%%%%%%%%%%%%%%%%%
\section{IR Another Example}
%%%%%%%%%%%%%%%%%%%%%%%%%%%%%
% Frame Open :: IR examples %
%%%%%%%%%%%%%%%%%%%%%%%%%%%%%
\frame{\frametitle{IR Example: \textbf{if (2$<$6) \{ PrintInt(3); \}}}
\begin{itemize}
%\item Consider the simple statement: \textit{if (2$<$6) $\{$ PrintInt(3); $\}$}.
\item IR is produced by scanning the AST recursively as follows:
\begin{itemize}
\item First, the condition subtree is scanned, producing the IR commands:
\begin{itemize}
\item \textbf{li  Temp\_74, 2}
\item \textbf{li  Temp\_75, 6}
\item \textbf{li  Temp\_76, 1}
\item \textbf{blt Temp\_74, Temp\_75, label\_cond\_end}
\item \textbf{li  Temp\_76, 0}
\item \textbf{label\_cond\_end}
\end{itemize}
\item Then, the if-father-node uses the temporary returned from
      its condition-son and wraps the IR commands produced by its
      body-son as follows:
\begin{itemize}
\item \textbf{beq Temp\_76, 0, label\_if\_end}
\item \textbf{li   Temp\_77, 3}
\item \textbf{call PrintInt( Temp\_77 )}
\item \textbf{label label\_if\_end}
\end{itemize}
\end{itemize}
\end{itemize}
%%%%%%%%%%%%%%%%%%%%%%%%%%%%%%%%%%%
% Frame Close :: Warm up examples %
%%%%%%%%%%%%%%%%%%%%%%%%%%%%%%%%%%%
%%%%%%%%%%%%%%%%%%%%%%%%%%%%%%%%%%%
% Frame Close :: Warm up examples %
%%%%%%%%%%%%%%%%%%%%%%%%%%%%%%%%%%%
}