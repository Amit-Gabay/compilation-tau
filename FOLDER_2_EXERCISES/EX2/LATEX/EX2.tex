%%%%%%%%%%%%%%%%%%
% DOCUMENT CLASS %
%%%%%%%%%%%%%%%%%%
\documentclass{article}

%%%%%%%%%%%%
% PACKAGES %
%%%%%%%%%%%%
\usepackage{hyperref}

%%%%%%%%%%%%%%%%%%
% BEGIN DOCUMENT %
%%%%%%%%%%%%%%%%%%
\begin{document}

%%%%%%%%%
% TITLE %
%%%%%%%%%
\title{Exercise 2}

%%%%%%%%%%%%%%%%%%%%%%%%%%%%%
% AUTHOR = COURSE NAME HERE %
%%%%%%%%%%%%%%%%%%%%%%%%%%%%%
\author{Compilation 0368:3133}

%%%%%%%%%%%%%%%%%%%%%%%%%%%%%%%
% DATE = SUBMISSION DATE HERE %
%%%%%%%%%%%%%%%%%%%%%%%%%%%%%%%
\date{Due 21/11/2018, before 07:00}

%%%%%%%%%
% TITLE %
%%%%%%%%%
\maketitle

%%%%%%%%%%%%%%%%%%%%%%%%%%%
% SECTION :: Introduction %
%%%%%%%%%%%%%%%%%%%%%%%%%%%
\section{Introduction}
We continue our journey of building a compiler
for the invented object oriented language Picasso.
In order to make this document self contained,
all the information needed to complete the second exercise is brought here.
%%%%%%%%%%%%%%%%%%%%%%%%%%%%%%%%%%%%%
% SECTION :: Programming Assignment %
%%%%%%%%%%%%%%%%%%%%%%%%%%%%%%%%%%%%%
\section{Programming Assignment}
The second exercise implements a \href{http://www2.cs.tum.edu/projects/cup/}{CUP} based
parser on top of your \href{http://jflex.de/}{JFlex} scanner from the exercise $1$.
The input for the parser is a single text file containing a Picasso program,
and the output is a (single) text file indicating whether the input program
is syntactically valid or not. In addition to that,
whenever the input program has correct syntax,
the parser should internally create the abstract syntax tree (AST).
Currently, the course repository contains a simple skeleton
parser, that indicates whether the input program has correct syntax,
and internally builds an AST for a small subset of Picasso.
As always, you are encouraged to work your way up from there,
but feel free to write the whole exercise from scratch if you want to.
Note also, that the AST will not be checked in exercise $2$.
It is needed for later phases (semantic analyzer and code generation)
but the best time to design and implement the AST is exercise $2$.
%%%%%%%%%%%%%%%%%%%%%%%%%%%%%%%%%
% SECTION :: The Picasso Syntax %
%%%%%%%%%%%%%%%%%%%%%%%%%%%%%%%%%
\section{The Picasso Syntax}
Table \ref{Table_CFG_Of_Picasso} specifies the context free grammar of Picasso.
You will need to feed this grammar to \href{http://www2.cs.tum.edu/projects/cup/}{CUP},
and make sure there are no shift-reduce conflicts.
The operator precedence is listed in Table
\ref{Table_Binary_Operators_Of_Picasso}. 
\begin{table}[h]
\centering
\begin{tabular}{ l c l }
%%%%%%%%%%%%%%%%%%%%%%%%%%%%%%%
Program  & $::=$ & dec$^{+}$ \\
%%%%%%%%%%%%%%%%%%%%%%%%%%%%%%%
\\
%%%%%%%%%%%%%%%%%%%%%%%%%%%%%%%%%%%%%%%%%%%%%%%%%%%%%%%%%%%%%%%%%%
dec      & $::=$ & varDec $|$ funcDec $|$ classDec $|$ arrayDec \\
%%%%%%%%%%%%%%%%%%%%%%%%%%%%%%%%%%%%%%%%%%%%%%%%%%%%%%%%%%%%%%%%%%
\\
%%%%%%%%%%%%%%%%%%%%%%%%%%%%%%%%%%%%%%%%%%%%%%%%%%
varDec   & $::=$ & ID ID $[$ ASSIGN exp $]$ ';' \\
         & $::=$ & ID ID ASSIGN newExp ';'      \\
%%%%%%%%%%%%%%%%%%%%%%%%%%%%%%%%%%%%%%%%%%%%%%%%%%
%%%%%%%%%%%%%%%%%%%%%%%%%%%%%%%%%%%%%%%%%%%%%%%%%%%%%%%%%%%%%%%%%%%%%%%%%%%
funcDec  & $::=$ & ID ID $'('$ $[$ ID ID $[$ ',' ID ID $]^{*}$ $]$ $')'$ %%
                   $'\{'$ stmt   $[$ stmt $]^{*}$ $'\}'$                 \\
%%%%%%%%%%%%%%%%%%%%%%%%%%%%%%%%%%%%%%%%%%%%%%%%%%%%%%%%%%%%%%%%%%%%%%%%%%%
%%%%%%%%%%%%%%%%%%%%%%%%%%%%%%%%%%%%%%%%%%%%%%%%%%%%%%%%%%%%%%%%%%%%%%%%%%%%%%%%%%%%%%%%%
classDec & $::=$ & CLASS ID $[$ EXTENDS ID $]$ $'\{'$ cField $[$ cField $]^{*}$ $'\}'$ \\
%%%%%%%%%%%%%%%%%%%%%%%%%%%%%%%%%%%%%%%%%%%%%%%%%%%%%%%%%%%%%%%%%%%%%%%%%%%%%%%%%%%%%%%%%
%%%%%%%%%%%%%%%%%%%%%%%%%%%%%%%%%%%%%%%%%%%%%%%%%
arrayDec & $::=$ & ARRAY ID $=$ ID $'['$ $']'$ \\
%%%%%%%%%%%%%%%%%%%%%%%%%%%%%%%%%%%%%%%%%%%%%%%%%
\\
%%%%%%%%%%%%%%%%%%%%%%%%%%%%%%%%%%%%%%%%%%%%%%%%%%%%%%%%%%%%%%%%%%%%%%%%%%%%%%%%%%%%
exp      & $::=$ & var                                                            \\
         & $::=$ & $'('$ exp $')'$                                                \\
         & $::=$ & exp BINOP exp                                                  \\
         & $::=$ & $[$ var '.' $]$ ID $'('$ $[$ exp $[$ ',' exp $]^{*}$ $]$ $')'$ \\
         & $::=$ & $['-']$ INT $|$ NIL $|$ STRING                                 \\
%%%%%%%%%%%%%%%%%%%%%%%%%%%%%%%%%%%%%%%%%%%%%%%%%%%%%%%%%%%%%%%%%%%%%%%%%%%%%%%%%%%%
\\
%%%%%%%%%%%%%%%%%%%%%%%%%%%%%%%%%%%%%%%%%
var      & $::=$ & ID                  \\
         & $::=$ & var '.' ID          \\
         & $::=$ & var $'['$ exp $']'$ \\
%%%%%%%%%%%%%%%%%%%%%%%%%%%%%%%%%%%%%%%%%
\\  
%%%%%%%%%%%%%%%%%%%%%%%%%%%%%%%%%%%%%%%%%%%%%%%%%%%%%%%%%%%%%%%%%%%%%%%%%%%%%%%%%%%%%%%%
stmt     & $::=$ & varDec                                                             \\
         & $::=$ & var ASSIGN exp ';'                                                 \\
         & $::=$ & var ASSIGN newExp ';'                                              \\
         & $::=$ & RETURN $[$ exp $]$ ';'                                             \\
         & $::=$ & IF $'('$ exp $')'$ $'\{'$ stmt $[$ stmt $]^{*}$ $'\}'$             \\
         & $::=$ & WHILE $'('$ exp $')'$ $'\{'$ stmt $[$ stmt $]^{*}$ $'\}'$          \\
         & $::=$ & $[$ var '.' $]$ ID $'('$ $[$ exp $[$ ',' exp $]^{*}$ $]$ $')'$ ';' \\
%%%%%%%%%%%%%%%%%%%%%%%%%%%%%%%%%%%%%%%%%%%%%%%%%%%%%%%%%%%%%%%%%%%%%%%%%%%%%%%%%%%%%%%%
\\
%%%%%%%%%%%%%%%%%%%%%%%%%%%%%%%%%%%%%%%%%%%%%%%%%%%%%%%
newExp   & $::=$ & NEW ID $|$ NEW ID $'['$ exp $']'$ \\
%%%%%%%%%%%%%%%%%%%%%%%%%%%%%%%%%%%%%%%%%%%%%%%%%%%%%%%
\\
%%%%%%%%%%%%%%%%%%%%%%%%%%%%%%%%%%%%%%%%
cField   & $::=$ & varDec $|$ funcDec \\
%%%%%%%%%%%%%%%%%%%%%%%%%%%%%%%%%%%%%%%%
%%%%%%%%%%%%%%%%%%%%%%%%%%%%%%%%%%%%%%%%%%%%%%%%%%%%%%%%%%%%%%%%%%%%%%%%%
BINOP    & $::=$ & $+$ $|$ $-$ $|$ $*$ $|$ $/$ $|$ $<$ $|$ $>$ $|$ $=$ \\
INT      & $::=$ & $[1-9][0-9]^{*}$ $|$ $0$                            \\
%%%%%%%%%%%%%%%%%%%%%%%%%%%%%%%%%%%%%%%%%%%%%%%%%%%%%%%%%%%%%%%%%%%%%%%%%
\\
\end{tabular}
\caption{
Context free grammar for the Picasso programming language.
\label{Table_CFG_Of_Picasso}}
\end{table}
\begin{table}[h]
\centering
\begin{tabular}{ |c|c|l|l| }
\hline
Precedence & Operator & Description & Associativity \\
\hline
\hline
%%%%%%%%%%%%%%%%%%%%%%%%%%%%%%%%%%%%%%%%%%%%%%%%%%%%%%%%
1          & $:=$            & assign         &       \\
%%%%%%%%%%%%%%%%%%%%%%%%%%%%%%%%%%%%%%%%%%%%%%%%%%%%%%%%
\hline
%%%%%%%%%%%%%%%%%%%%%%%%%%%%%%%%%%%%%%%%%%%%%%%%%%%%%%%%
2          & $=$             & equals         & left  \\
%%%%%%%%%%%%%%%%%%%%%%%%%%%%%%%%%%%%%%%%%%%%%%%%%%%%%%%%
\hline
%%%%%%%%%%%%%%%%%%%%%%%%%%%%%%%%%%%%%%%%%%%%%%%%%%%%%%%%
3          & $<,>$           &                & left  \\
%%%%%%%%%%%%%%%%%%%%%%%%%%%%%%%%%%%%%%%%%%%%%%%%%%%%%%%%
\hline
%%%%%%%%%%%%%%%%%%%%%%%%%%%%%%%%%%%%%%%%%%%%%%%%%%%%%%%%
4          & $+,-$           &                & left  \\
%%%%%%%%%%%%%%%%%%%%%%%%%%%%%%%%%%%%%%%%%%%%%%%%%%%%%%%%
\hline
%%%%%%%%%%%%%%%%%%%%%%%%%%%%%%%%%%%%%%%%%%%%%%%%%%%%%%%%
5          & $*,/$           &                & left  \\
%%%%%%%%%%%%%%%%%%%%%%%%%%%%%%%%%%%%%%%%%%%%%%%%%%%%%%%%
\hline
%%%%%%%%%%%%%%%%%%%%%%%%%%%%%%%%%%%%%%%%%%%%%%%%%%%%%%%%
6          & $[$             & array indexing &       \\
%%%%%%%%%%%%%%%%%%%%%%%%%%%%%%%%%%%%%%%%%%%%%%%%%%%%%%%%
\hline
%%%%%%%%%%%%%%%%%%%%%%%%%%%%%%%%%%%%%%%%%%%%%%%%%%%%%%%%
7          & $($             & function call  &       \\
%%%%%%%%%%%%%%%%%%%%%%%%%%%%%%%%%%%%%%%%%%%%%%%%%%%%%%%%
\hline
%%%%%%%%%%%%%%%%%%%%%%%%%%%%%%%%%%%%%%%%%%%%%%%%%%%%%%%%
8          & $.$     & field access   & left          \\
%%%%%%%%%%%%%%%%%%%%%%%%%%%%%%%%%%%%%%%%%%%%%%%%%%%%%%%%
\hline
\end{tabular}
\caption{
Binary operators of Picasso along with their associativity and precedence.
$1$ stands for the lowest precedence, and $9$ for the highest.
\label{Table_Binary_Operators_Of_Picasso}}
\end{table}
%%%%%%%%%%%%%%%%%%%%
% SECTION :: Input %
%%%%%%%%%%%%%%%%%%%%
\section{Input}
The input for this exercise is a single text file, the input Picasso program.
%%%%%%%%%%%%%%%%%%%%%
% SECTION :: Output %
%%%%%%%%%%%%%%%%%%%%%
\section{Output}
The output is a \textit{single} text file that contains a \textit{single} word.
Either OK when the input program has correct syntax,
or otherwise ERROR(\textit{location}), where \textit{location}
is the line number of the \textit{first} error that was encountered.
%%%%%%%%%%%%%%%%%%%%%%%%%%%%%%%%%%%%
% SECTION :: Submission Guidelines %
%%%%%%%%%%%%%%%%%%%%%%%%%%%%%%%%%%%%
\section{Submission Guidelines}
The skeleton code for this exercise resides (as usual)
in subdirectory EX2 of the course repository.
You need to add the relevant derivation rules and AST constructors.
COMPILATION/EX2 should contain a makefile building your source files to a
runnable jar file called PARSER (note the lack of the .jar suffix).
Feel free to use the makefile supplied in the course repository,
or write a new one if you want to. 
Before you submit, make sure that your exercise compiles and runs
on the school server: $nova.cs.tau.ac.il$.
This is the formal running environment of the course.
\paragraph{Execution parameters}
PARSER receives $2$ input file names:\\ \\
InputPicassoProgram.txt\\
OutputStatus.txt

\end{document}
